\section[Communiquer]{Communiquer entre ordinateur} % Prendre le modèle TCP/IP en 4 couches, qu'on simplifie en 3 le but n'est pas de faire un cours sur le sujet https://fr.wikipedia.org/wiki/Suite_des_protocoles_Internet
\begin{frame}
	\begin{itemize}
		\item Un réseau informatique est ensemble d'équipement reliés entre eux pour échanger de l'information.
		\item Le modèle \textbf{OSI}\footnote{\forme{Open System Interconnexion}} propose un standard en \textbf{7 couche}.
		\item Mais on simplifiera ici à \textbf{3 couches}.
		
	\end{itemize}
\end{frame}
\subsection{Un support physique}

\begin{frame}
	\begin{itemize}
		\item Deux humains qui communiquent utilise un support physique pour le faire : 			\begin{itemize}
				\item Son. 
				\item Signaux de fumées.
				\item Gestes.
				\item etc.
			\end{itemize}
		\item Les ordinateurs de même:
			\begin{itemize}
				\item Câbles métalliques, notamment cuivres.
				\item Ondes électromagnétiques (Wifi, 3G etc.).
				\item Fibre optique. % Note : la lumière est une onde élèctromagnétique de fréquence particulière, mais passons.
			\end{itemize} 
		\item Il existe des \textbf{normes} décrivant l'usage de ces supports.
		\item Notamment les normes \textbf{ethernet}.
	\end{itemize}
\end{frame}
\subsection{Mettre en réseau}
\begin{frame}
	\begin{itemize}
		\item Pour s'avoir à qui on écrit on utilise une adresse.
		\item La poste a des règles internes d'acheminement du courrier.
		\item Le système le plus répandu aujourd'hui est \textbf{Internet Protocol} (1980).
	\end{itemize}
\end{frame}

\begin{frame}
	\begin{itemize}
		\item On peut utiliser IP sans Internet : réseaux locaux.
		\item L'interconnexion des \textbf{réseaux locaux} forme \textbf{Internet}.
		\item IP se base sur notion de \textbf{paquets}:
			\begin{itemize}
				\item On divise les données en paquets.
				\item On ne sait pas d'avance par où passe les paquets.
				\item Se caractérise notamment par l'aspect \textbf{décentralisé}.
			\end{itemize}
		\item \formeenglish{Internet Corporation for Assigned Names and Numbers} attribut les adresses IP sur Internet:
			\begin{itemize}
				\item IPv4 : \num{4,3 94 967 296} : épuisées.
				\item IPv6 : \num{3,4e38} : \SI{3}{\percent}?
			\end{itemize}
	\end{itemize}

\end{frame}

\begin{frame}
	\begin{itemize}
		\item Pour éviter de retenir les adresses IP on a inventé (1983) le \textbf{nom de domaines}.
		\item Les domaines fonctionnent de manière hiérarchiques.
		\begin{description}
			\item[Premier niveau (TLD)] : extensions (\url{.fr}, \url{.org}).
			\item[Second niveau] : domaines (\url{wikipedia.org}, \url{unil.ch}).
			\item[Troisième niveau] : sous-domaine (\url{fr.wikipedia.org}).
			\item etc.
		\end{description}
	\end{itemize}
\end{frame}

\subsection{Que fait-on?}

\begin{frame}
	\begin{itemize}
		\item Courriers électroniques.
		\item Échanges de fichiers.
		\item Messageries instantanées.
		\item Le plus fameux \textbf{le World Wide Web/la Toile} (1990).
	\end{itemize}
\end{frame}


\begin{frame}
	\begin{itemize}
		\item Mettre à disposition de l'information.
		\item Relier de l'information sur la base de liens \textbf{hypertextes}.
		\item Pas de système centralisé.
	\end{itemize}
\end{frame}

\begin{frame}
	\begin{itemize}
		\item Une page web est envoyée à un \formeenglish{client} web par un \formeenglish{serveur} via \formeenglish{HyperText Transfer Protocol} (HTTP).
		\item Tout ordinateur est un serveur potentiel.
		\item Des \formeenglish{hébergeurs} proposent des  ordinateurs spécialisés comment serveur HTTP.
	\end{itemize}
\end{frame}

\begin{frame}
	Une page web :
	\begin{itemize}
		\item Localisable par \formeenglish{Uniform Resource Locator} (URL).
		\item Décrite en \formeenglish{HyperText Markup Language} (HTML).
		\item Mise en forme via \formeenglish{Cascading Style Sheets} (CSS).
		\item Complétée par du multimédia.
		\item Animée via \formeenglish{JavaScript}.
	\end{itemize}

\end{frame}

\begin{frame}
	Deux types :
	\begin{itemize}
		\item Statiques.
		\item Dynamiques.
		\begin{itemize}
			\item Logiciel de forums.
			\item Logiciel de Wiki.
			\item Logiciels de gestion de contenu (SPIP, WordPress, Drupal)
			\item etc.
		\end{itemize} 
	\end{itemize}
\end{frame}
