\section[Contrôler]{Qui contrôle quoi?}

\subsection{Une affaire d'humains}


\begin{slide}
	\begin{itemize}
		\item Des humains qui créent l'informatique.
		\item Des humains qui utilisent l'informatique.
		\item Des humains qui ont un pouvoir sur d'autres.
	\end{itemize}
\end{slide}

\begin{slide}

	
	\begin{itemize}
		\item Structurer
		\item Manipuler
		\item Échanger
	\end{itemize}
	

\end{slide}


\subsection{Formats fermés contre formats ouverts}

\begin{slide}
	\begin{itemize}
	\item Importance de pouvoir lire un fichier:
		\begin{itemize}
			\item Sur tout type de support.
			\item À tout moment.
		\end{itemize}
	\item Deux outils:
		\begin{itemize}
			\item Formats ouverts : spécifications connues.
			\item Formats standardisés : spécifications standardisées.
		\end{itemize}
	\end{itemize}
\end{slide}
\subsection{Logiciels propriétaires contre logiciels libres} % Faire une secion

\begin{slide}
	Logiciels distribuables sous deux formes :
	\begin{itemize}
		\item \textbf{Binaire} : compréhensible par la machine.
		\item \textbf{Code source} : compréhensible par l'humain.
	\end{itemize}

\end{slide}

\begin{slide}
	\begin{itemize}
		\item \textbf{Logiciel libre} (\formeenglish{Free Software Fondation}, 1986+). 4 libertés
		\begin{enumerate}
		\setcounter{enumi}{-1}
			\item Exécuter le programme, pour tous besoins.
			\item Étudier et modifier le programme.
			\item Distribuer le programme.
			\item Distribuer les versions modifiées du programme.
		\end{enumerate}
		\item Tendance libertaire
	\end{itemize}

\end{slide} 

\begin{slide}
	\begin{itemize}
		\item \textbf{Logiciel open source} (OSI, 1998+). 10 critères
			\begin{enumerate}
				\item Libre distribution.
				\item Accès au code source.
				\item Modification.
				\item Distribution des versions dérivés.
				\item Neutralité vis à vis des  personnes.
				\item Neutralité vis à vis des usages.
				\item Distribution de la licence complète avec le logiciel.
				\item Pas de licence spécifique à un produit.
				\item Pas de licence intervenant sur d'autres logiciels.
				\item Neutralité technologique.
			\end{enumerate}
		\item Tendance libertarienne
	\end{itemize}
\end{slide}

\begin{slide}
	\begin{itemize}
		\item La plupart des logiciels libres sont aussi open source (et vice-versa).
		\item La clause de \textbf{copyleft} est plus courante dans le monde du libre.
		\item Le libre constitue une grande part de nos services internet.
		\item Le monde du libre s'est étendu à d'autres domaines (art, littérature etc.).
	\end{itemize}
\end{slide}

\begin{slide}
	\begin{itemize}
		\item Organismes définissent des protocoles réseaux.
		\item Qui contrôlent ces organismes?
		\item Qui contrôle la distribution d'IP publique et de domaine?
	\end{itemize}
\end{slide}
