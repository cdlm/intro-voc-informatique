\section{Manipuler les données}
\subsection{Encore du binaire}

\begin{slide}
	\begin{itemize}
		\item[1950] Premier ordinateur à transitors.
		\item[1970] Premier ordinateur à microprocesseur. % Actuellement (2012, de l'ordre du milliard), gravé de l'ordre de de la dizaine de nanomètre, soit 10 000em de cheveux
	\end{itemize}
\end{slide}

\begin{slide}
	\begin{itemize}
		\item Le processeur traite les données.
		\item Il est composé de plusieurs transitors, avec des fonctions spécalisés.
		\item Il possède un jeu d'instruction spécifique.
		\item Chaque instruction est codée en binaire.
	\end{itemize}
\end{slide}

\subsection{Des langages de différents niveaux}

\begin{slide}
	\begin{itemize}
		\item Des langages de \textbf{très bas niveau}, dépendant du processeur:
		\begin{itemize}
			\item Binaire.
			\item Assembleur.
		\end{itemize}
		\item Des langages \textbf{compilés} de bas niveau:
		\begin{itemize}
			\item Basic
			\item C (et ses dérivés [C++,C\# etc.]?)
		\end{itemize}
		\item Des langages de haut \textbf{interprétés} de haut niveau:
		\begin{itemize}
			\item Java
			\item PHP
			\item Python
		\end{itemize}
	\end{itemize}
\end{slide}

\subsection{Des paradigmes différents}
\begin{slide}
	\begin{itemize}
		\item Impératif.
		\item Fonctionnel.
		\item Déclaratif.
		\item Par contrainte.
		\item Orienté objet.
		\item etc.
	\end{itemize}
\end{slide}

\subsection{Un exemple en langage impératif}
\begin{slide}
	\begin{exampleblock}{Ici Python}
		\inputminted{python}{exemples-manipulation/python.py}
	\end{exampleblock}
\end{slide}

\subsection{Des langages selon les besoins}
\begin{slide}
	\begin{itemize}
		\item Contexte d'utilisation.
		\item Type de données à manipuler.
		\item Performance.
		\item Simplicité et lisibilité du code.
		\item Portabilité.
		\item Habitude… 
	\end{itemize}
\end{slide}

\subsection{Précision de vocabulaire}
\begin{slide}
	\begin{itemize}
		\item[Algorithme] : série d'instructions à fournir à l'ordinateur.
		\item[Programme] : mise en forme numérique de ces instructions. 
		\item[Fonction] : unité logique dans un programme.
		\item[Script] : programme en langage interprété.
		\item[Logiciel] : suite de programme et d'outils pour le faire fonctionner (images, réglages etc.)
		\item[Application] : logiciel utilisé pour réaliser une activité.
		\item En pratique \textbf{programme},\textbf{logiciel} et \textbf{application} sont souvent interchangeables.
	\end{itemize}
\end{slide}