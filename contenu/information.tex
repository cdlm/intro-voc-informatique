\section[Traiter]{Traiter l'information}

\subsection{Informatique}

\begin{slide}
	\begin{itemize}
	\item Tous les jours nous manipulons de l'information pour accomplir des tâches.
	\item Exemple : établir une liste de course pour la cuisine de la semaine.
		
		\begin{itemize}
			\item Taille du frigo et des espaces de stockages.
			\item Nombre de personnes.
			\item Budget.
			\item Goûts culinaire, adaptés aux circonstances.
			\item etc.
		\end{itemize}
	\item À partir de là on établit une liste (implicite ou explicite) de course
	
	\end{itemize}
\end{slide}

\begin{slide}

Deux idées phares de l'informatique:
	\begin{enumerate}
		\item On peut systématiser le traitement des données.
		\item On peut automatiser le traitement des données.
	\end{enumerate}
\end{slide}

\begin{slide}

	\note[item]{\item De ces deux idées, c'est la première la plus importante, qui définit ce qu'est l'informatique \textbf{scientifique}.}
	
	
	\begin{block}{Informatique}
	Science du traitement systématique de l'information.
	\end{block}
	
	\note[item]{\textbf{En pratique}, l'informatique utilise des machines pour traiter l'information mais:}
	
	\begin{exampleblock}{Citation apocryphe (?) attribuée à Edsger Dijkstra}% Voir https://en.wikiquote.org/wiki/Computer_science#Disputed pour l'attribution
	Computer Science is no more about computers than astronomy is about telescopes.\\
	
	L'informatique n'est pas plus la sciences des ordinateurs que l'astronomie n'est celle des télescopes
	\end{exampleblock}
	
\end{slide}

\begin{slide}
	\begin{block}{Informaticien}<.->
		\begin{itemize}
			\item Conçoit comment structurer des données.
			\item Conçoit comment les manipuler de manière systématique : définit des \textbf{algorithmes}.
		\end{itemize}
	\end{block}
\end{slide}
\subsection{Numérique et digital}

\begin{slide}

	\begin{itemize}
		\item Pour permettre l'automatisation du traitement des données par des machines on les ramène à des \emph{nombres}.
		\item D'où les termes:
			\begin{itemize}
				\item \forme{Numérique}
				\item \forme{Digital} (\formeenglish{digit} = \forme{chiffre})
			\end{itemize}
		\item Mais \textbf{en pratique} rare sont les informaticiens qui traitent les données comme des nombres. 
	\end{itemize}
\end{slide}

\begin{slide}
	\begin{itemize}
		\item \forme{Humanités numériques} : traduction de \forme{digital humanities}.
		\begin{itemize}
			\item Respecte le vocabulaire du français.
			\item Respecte la sémantique de l'anglais.
		\end{itemize}
		
		\item \forme{Humanités digitales} : transposition de \forme{digital humanities}.
			\begin{itemize}
				\item Ne respecte le vocabulaire du français.
				\item Ne respecte pas la sémantique de l'anglais.
				\item \textbf{Mais} \enquote{nous entrons en contact avec le monde digital avec nos doigts} (C. Clivaz). %http://claireclivaz.hypotheses.org/114
				\item \textbf{Néammoins}:
				\begin{itemize}
					\item Concerne non pas l'\textbf{informatique} mais les \textbf{ordinateurs}.
					\item N'est pas nouveau.
					\item Peut s'appliquer à l'étude sociologique de l'utilisation de l'informatique mais difficilement à l'utilisation de l'informatique dans le domaine des SHS.
				\end{itemize}
			\end{itemize}
		\item \textbf{Proposition} : \forme{humanités informatisées} ou \forme{humanité informationalisées}.
	\end{itemize}
\end{slide}
\subsection{Ordinateurs, smartphones et autres machines}

\begin{slide}
	\begin{itemize}
		\item \formeenglish{computer} : \forme{calculateur}.
		\item IBM France invente \forme{ordinateur}, inspiré de \forme{ordonnateur} (désuet).
	\end{itemize}
\end{slide}

\begin{slide}
	\begin{itemize}
		\item Un ordinateur est une réalisation concrète d'une \textbf{machine de Turing universelle} (1936):
			\begin{itemize}
				\item Un ruban divisé en cases contenant de l'information notés sous forme d'alphabet (par ex : 0/1).
				\item Une tête de lecture/écriture :
					\begin{itemize}
						\item Lit ou écrit sur une case à la fois.
						\item Peut se déplacer sur le ruban, vers la gauche ou la droite.
					\end{itemize}
				\item Un registre d'état mémorisant l'état courant de la machine.
				\item Une table d'actions qui détermine :
					\begin{itemize}
						\item Qu'écrire sur la case courante.
						\item Dans quel sens déplacer la tête de lecture.
						\item Comment modifier le registre d'état.
					\end{itemize}
				\item	Et ce en fonction de :
					\begin{itemize}
						\item L'état courant.
						\item Le contenu de la case courante.
					\end{itemize}
				\item Puisque la machine est \textbf{universelle} la table d'action est elle même écrite sur le ruban.
			\end{itemize}
	
		\item C'est donc la réalisation du second postulat de l'informatique : on peut automatiser une série d'opération de transformation de données.
	\end{itemize}
\end{slide}

\begin{slide}
	\begin{itemize}
		\item Plus simplement un ordinateur est une machine qui peut :
			\begin{itemize}
				\item Lire des \textbf{données}. 
				\item Manipuler les données en fonctions d'\textbf{instructions} :
					\begin{itemize}
						\item Modifiables par l'utilisateur.
						\item Non linéaires (possibilité du \textbf{SI}).
					\end{itemize}
				\item Fournir le résultat de ces manipulations. 
			\end{itemize}
		\item Techniquement les choix sont multiples:
			\begin{itemize}
				\item Pour la manipulation des données:
					\begin{itemize}
						\item Cables que l'on déplace et tubes à vides (premiers ordinateurs).
						\item Transitors, aujourd'hui regroupés en micro-processeurs (ordinateurs modernes)
						\
					\end{itemize}
				\item Pour la lecture et l'écriture des données des données:
					\begin{itemize}
						\item Cartes perforées.
						\item Clavier
						\item Mémoire électroniques (vives et mortes)
						\item Écran (tactiles ou non)
						\item etc.
					\end{itemize}
			\end{itemize}
	\end{itemize}
\end{slide}
\begin{slide}
	\begin{itemize}
		\item Micro-ordinateurs de bureau, ordinateurs portables, tablettes, smartphones etc.:
			\begin{itemize}
				\item Sont tous des ordinateurs.
				\item Portent des noms différents pour des raisons commerciales.
				\item Peuvent, pour des raisons juridico-commerciales, être limités dans la possibilité \emph{effective} de modifier les instructions (ex : \textbf{Apple Store}).
			\end{itemize}
	\end{itemize}
\end{slide}